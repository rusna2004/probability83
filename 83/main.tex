\documentclass{article}[]
\usepackage{amsmath, amsfonts, amssymb}
\usepackage{float}
\begin{document}
\title{Question 12.13.3.83}
\author{Rusna Shaikh\\EE22BTECH11048}
\date{}
\maketitle{}
\providecommand{\pr}[1]{\ensuremath{\Pr\left(#1\right)}}
\providecommand{\prt}[2]{\ensuremath{p_{#1}^{\left(#2\right)} }}        % own macro for this question
\providecommand{\qfunc}[1]{\ensuremath{Q\left(#1\right)}}
\providecommand{\sbrak}[1]{\ensuremath{{}\left[#1\right]}}
\providecommand{\lsbrak}[1]{\ensuremath{{}\left[#1\right.}}
\providecommand{\rsbrak}[1]{\ensuremath{{}\left.#1\right]}}
\providecommand{\brak}[1]{\ensuremath{\left(#1\right)}}
\providecommand{\lbrak}[1]{\ensuremath{\left(#1\right.}}
\providecommand{\rbrak}[1]{\ensuremath{\left.#1\right)}}
\providecommand{\cbrak}[1]{\ensuremath{\left\{#1\right\}}}
\providecommand{\lcbrak}[1]{\ensuremath{\left\{#1\right.}}
\providecommand{\rcbrak}[1]{\ensuremath{\left.#1\right\}}}
\newcommand{\sgn}{\mathop{\mathrm{sgn}}}
\providecommand{\abs}[1]{\left\vert#1\right\vert}
\providecommand{\res}[1]{\Res\displaylimits_{#1}} 
\providecommand{\norm}[1]{\left\lVert#1\right\rVert}
%\providecommand{\norm}[1]{\lVert#1\rVert}
\providecommand{\mtx}[1]{\mathbf{#1}}
\providecommand{\mean}[1]{E\left[ #1 \right]}
\providecommand{\cond}[2]{#1\middle|#2}
\providecommand{\fourier}{\overset{\mathcal{F}}{ \rightleftharpoons}}
\newenvironment{amatrix}[1]{%
  \left(\begin{array}{@{}*{#1}{c}|c@{}}
}{%
  \end{array}\right)
}
%\providecommand{\hilbert}{\overset{\mathcal{H}}{ \rightleftharpoons}}
%\providecommand{\system}{\overset{\mathcal{H}}{ \longleftrightarrow}}
	%\newcommand{\solution}[2]{\textbf{Solution:}{#1}}
\newcommand{\solution}{\noindent \textbf{Solution: }}
\newcommand{\cosec}{\,\text{cosec}\,}
\providecommand{\dec}[2]{\ensuremath{\overset{#1}{\underset{#2}{\gtrless}}}}
\newcommand{\myvec}[1]{\ensuremath{\begin{pmatrix}#1\end{pmatrix}}}
\newcommand{\mydet}[1]{\ensuremath{\begin{vmatrix}#1\end{vmatrix}}}
\newcommand{\myaugvec}[2]{\ensuremath{\begin{amatrix}{#1}#2\end{amatrix}}}
\providecommand{\rank}{\text{rank}}
\providecommand{\pr}[1]{\ensuremath{\Pr\left(#1\right)}}
\providecommand{\qfunc}[1]{\ensuremath{Q\left(#1\right)}}
	\newcommand*{\permcomb}[4][0mu]{{{}^{#3}\mkern#1#2_{#4}}}
\newcommand*{\perm}[1][-3mu]{\permcomb[#1]{P}}
\newcommand*{\comb}[1][-1mu]{\permcomb[#1]{C}}
\providecommand{\qfunc}[1]{\ensuremath{Q\left(#1\right)}}
\providecommand{\gauss}[2]{\mathcal{N}\ensuremath{\left(#1,#2\right)}}
\providecommand{\diff}[2]{\ensuremath{\frac{d{#1}}{d{#2}}}}
\providecommand{\myceil}[1]{\left \lceil #1 \right \rceil }
\newcommand\figref{Fig.~\ref}
\newcommand\tabref{Table~\ref}
\newcommand{\sinc}{\,\text{sinc}\,}
\newcommand{\rect}{\,\text{rect}\,}
%%
%	%\newcommand{\solution}[2]{\textbf{Solution:}{#1}}
%\newcommand{\solution}{\noindent \textbf{Solution: }}
%\newcommand{\cosec}{\,\text{cosec}\,}
%\numberwithin{equation}{section}
%\numberwithin{equation}{subsection}
%\numberwithin{problem}{section}
%\numberwithin{definition}{section}
%\makeatletter
%\@addtoreset{figure}{problem}
%\makeatother

%\let\StandardTheFigure\thefigure
\let\vec\mathbf



83. Which one is not a requirement of a binomial distribution?
\begin{enumerate}
\item There are 2 outcomes for each trial
\item There is a fixed number of trials
\item The outcomes must be dependent on each other
\item The probability of success must be the same for all the trials
\end{enumerate}
\solution
\begin{enumerate}
\item
{
In a binomial distribution, each trial has two possible outcomes, typically labeled as "success" (S) and "failure" (F). The probability of success is denoted as "p," and the probability of failure is denoted as "q," where q = 1 - p.

Example:

Consider flipping a fair coin 5 times and counting the number of times you get heads
\begin{enumerate}
\item Probability of success(heads): p=0.5
\item Probability of failure(tails): q=1-p=0.5
\item Number of trials: n=5
\end{enumerate}
The pmf of this binomial distribution is given by:
\begin{align}
\Pr(X=k) = \comb{n}{r} p^{k} q^{n-k} \\
\text{For k=0,1,2,3,4,5}\\
\Pr(X=0) = \comb{5}{0} (0.5)^{0} (0.5)^{5} = \frac{1}{32}\\
\Pr(X=1) = \comb{5}{1} (0.5)^{1} (0.5)^{4} = \frac{5}{32}\\
\Pr(X=2) = \comb{5}{2} (0.5)^{2} (0.5)^{3} = \frac{10}{32}\\
\Pr(X=3) = \comb{5}{3} (0.5)^{3} (0.5)^{2} = \frac{10}{32}\\
\Pr(X=4) = \comb{5}{4} (0.5)^{4} (0.5)^{1} = \frac{5}{32}\\
\Pr(X=5) = \comb{5}{5} (0.5)^{5} (0.5)^{0} = \frac{1}{32}
\end{align}

As we can see, this binomial distribution has two possible outcomes for each trial.
}
\item
{
In a binomial distribution, there must be a predetermined fixed number of trials, denoted as "n."
From the example used above, we can see that the number of trials are fixed.
}
\item
{
This is not a requirement of the binomial distribution. The outcomes of each trial must be independent of each other.

Example:

Consider rolling a fair six-sided die 10 times and counting the number of times you roll a 4 (success) in those 10 rolls. Each roll of the die is independent of the others, meeting the requirement of independence.
}
\item
{
In a binomial distribution, the probability of success (p) must remain constant across all trials.

Example: 

Suppose you are drawing cards from a well-shuffled standard deck of 52 cards with replacement and counting the number of times you draw a spade (success) in 8 draws. The probability of drawing a spade (p) is always 1/4 (since there are 13 spades in a deck of 52 cards), and this probability remains constant for each draw, meeting the requirement.
}
\end{enumerate}
\end{document}
